% Code styles are defined in code-styles.tex
\DeclareNewFloatType{algorithm}{name=Algorithm, placement=htbp, within=section}
\DeclareNewFloatType{codefragment}{name=Code fragment, placement=htbp,within=section}

\floatsetup[algorithm]{style=plaintop}
\floatsetup[codefragment]{style=plaintop}
%\floatsetup[table]{style=plaintop}				% Optional

% EXAMPLES BELOW EXAMPLES BELOW EXAMPLES BELOW EXAMPLES BELOW 

% Example algorithm:
% ====================================

%\begin{algorithm}
%\caption{Compute $N!$.}
%\label{alg:myalgorithm}
%\begin{lstlisting}[style=pseudoSt]
%$z \gets 1$ 
%for $k\in[1,\dotsc,N]$ do
%    $z \gets z\cdot k$
%return $z$
%\end{lstlisting}
%\end{algorithm}

% Example code fragment using inline code:
% ====================================

%\begin{codefragment}
%\caption{Compute $N!$, C++ implementation.}
%\label{alg:myprogram}
%\begin{lstlisting}[style=matlabSt]
%a = 1
%\end{lstlisting}
%\end{codefragment}

% Example code fragment using an imported file:
% ====================================

%\begin{codefragment}
%\caption{Compute $N!$, C++ implementation.}
%\label{alg:myprogram}
%\lstinputlisting[style=matlabSt]{enter path to code file}
%\end{codefragment}

% Include a List of Algorithms/Code fragments:
% ====================================

% \listof{algorithm}{List of Algorithms}
% \listof{codefragment}{List of Code fragments}